\documentclass[11pt,a4paper,oneside]{book}
%\usepackage[utf8]{inputenc} 	
\usepackage{a4wide}                     % Iets meer tekst op een bladzijde
\usepackage[english,dutch]{babel}               % Voor nederlandstalige hyphenatie (woordsplitsing)
\usepackage{graphicx}                   % Om figuren te kunnen verwerken
\usepackage[small,bf,hang]{caption}     % Om de captions wat te verbeteren
\usepackage{xspace}                     % Magische spaties na een commando
\usepackage[latin1]{inputenc}           % Om niet ascii karakters rechtstreeks te kunnen typen
\usepackage{subcaption}					% 2 figuren naast mekaar, met elk een caption
\usepackage[bottom]{footmisc}			% footnotes forceren naar bottom
\usepackage{nomencl}					%afkortingen
\usepackage{fancyhdr}                   % Voor fancy headers en footers
\makenomenclature %update bij every run
\renewcommand{\nomname}{Lijst van afkortingen}

\setlength{\parindent}{0cm}             % Inspringen van eerste lijn van paragrafen is niet gewenst.
\graphicspath{{figuren/}}               % De plaars waar latex zijn figuren gaat halen.

\makeindex                              % Om een index te genereren.

% De headers die verschijnen bovenaan de bladzijden, herdefinieren:
\pagestyle{fancy}                       % Om aan te geven welke bladzijde stijl we gebruiken.
\fancyhf{}                              % Resetten van al de fancy settings.
\renewcommand{\headrulewidth}{0pt}      % Geen lijn onder de header. Zet dit op 0.4pt voor een mooie lijn.
\fancyhf[HL]{\nouppercase{\textit{\leftmark}}} % Links in de header zetten we de leftmark,
\fancyhead[HR]{\thepage}                % Rechts in de header het paginanummer.

%commands
\newcommand{\npar}{\par \vspace{2.3ex plus 0.3ex minus 0.3ex}}

% Nieuw commando om figuren in te voegen. Gebruik:
% \mijnfiguur[H]{width=5cm}{bestandsnaam}{Het bijschrift bij deze figuur}
\newcommand{\mijnfiguur}[4][H]{            % Het eerste argument is standaar `ht'. op H zetten voor HIER EN NERGENS ANDERS
    \begin{figure}[#1]                      % Beginnen van de figure omgeving
        \begin{center}                      % Beginnen van de center omgeving
            \includegraphics[#2]{#3}        % Het eigenlijk invoegen van de figuur (2: opties, 3: bestandsnaam)
            \caption{#4\label{#3}}          % Het bijschrift (argument 4) en het label (argument 3)
        \end{center}
    \end{figure}
}
    
    
    
\begin{document}


\title{Theorie vragen BSIII - windows 2014}
\author{Andreas De Lille}
\renewcommand{\today}{Augustus 2014}
\maketitle

\frontmatter
%\setcounter{tocdepth}{5}
% De lijnen van de inhoudsopgave iets dichter op elkaar, niet echt nodig voor de thesis, maar 
% voor dit werk kregen we anders een laatste bladzijde met 3 items op.
\renewcommand{\baselinestretch}{1.08} 	% De interlinie afstand wat vergroten.
\small\normalsize                       % Nodig om de baselinestretch goed te krijgen.
\tableofcontents
\renewcommand{\baselinestretch}{1.2} 	% De interlinie afstand wat vergroten.
\small\normalsize                       % Nodig om de baselinestretch goed te krijgen.

\mainmatter
%reeks 1
\part{Mondeling - Reeks 1}
\chapter{Structuur van Active Directory gegevens}

\section{Bespreek de diverse namen die alle Active Directory\\ objecten identificeren. (2.2.1)}

\textbf{Naamgeving van object}\\
De namen zijn logisch en hi\"erarchisch opgebouwd.\\
Volgende vier namen zijn steeds beschikbaar.

\begin{enumerate}

\item \textbf{RDN - Relative distinguished name}
\begin{itemize}
\item Voorbeeld: cn = beelzebub
\item Is uniek binnen zijn container.
\item Denk aan absoluut path (DN) vs. relatief filepath (RDN).
\item Wordt opgeslagen in het cn attribuut van het object.
\end{itemize}

\item \textbf{DN - Distinguished name}
\begin{itemize}
\item Voorbeeld: cn = beelzebub, ou= iii, ou=hogent, ou=be (cn = common name, ou = organisational unit)
\item Attributed naming, verschillende attribuut=waarde koppels
\item Afgeleid van alle container object waarvan het object hi\"erarchisch deel uitmaakt.
\item Uniek over het hele domein.
\item Denk aan absoluut path (DN) vs. relatief filepath (RDN).
\item Wordt opgeslagen in het distinguishedName attribuut van het object.
\end{itemize}
\clearpage
\item \textbf{CN - canonieke naam}
\begin{itemize}
\item !! Niet hetzelfde als de cn van hierboven. Hier cn = canonieke naam ; hierboven cn = common name als waarde van een DN)
\item Voorbeeld: hogent.be/iii/beelzebub
\item Samengesteld uit de DN, geeft de DN op een eenvoudigere manier weer.
\item De meeste hulpmiddelen in active directory tonen de canonieke naam.
\item Wordt opgeslagen in het canonicalName attribuut van het object. (en dus niet in het cn attribuut)
\end{itemize}

\item \textbf{GUID - global unique identifier}
\begin{itemize}
\item Globaal uniek (zelfs in tijd) getal van 128 bits.
\item Kan en wordt nooit gewijzigd.
\item Wordt opgeslagen in het objectGUID attribuut van het object.
\item Wordt gegenereerd en toegewezen bij het aanmaken van het object.
\end{itemize}

\end{enumerate}

\section{Wat zijn SPN objecten ? Bespreek de aanvullende naamgeving voor deze objecten. (2.2.2)}

\begin{enumerate}

\item \textbf{SPN - Security Principal Objects}
\begin{itemize}
\item Doel: SPN of Security Principal Objects zijn Active Directory objecten die gebruikt worden om toegang te verlenen tot domeinbronnen. 
\item Zijn van toepassing op computers, gebruikersrekeningen en domeinen.
\end{itemize}

\item \textbf{SID - Security ID}
\begin{itemize}
\item Zijn net als guids uniek in tijd; wanneer een object verwijderd en vervolgens terug aangemaakt wordt, zal het een andere SPN krijgen. Hierdoor kan een object nooit machtigingen van een oude account behouden.
\item Opgeslagen in het objectSid kenmerk
\item Men maakt gebruik van SIDs naast GUIDs om compabiliteitsredenen.
\item hi\"erarchische string getallen gescheiden door koppeltekens bijvoorbeeld S-1-5-x-y-z-500. Hierbij is S-1-5 een standaard prefix bestaande uit een revision level en een authority identifier. X,y en z zijn 32bit getallen die specifiek zin voor het domain, (Domain Subauthority Identifier), 500 is een relatieve ID (RID) dat naar het feitelijke object verwijst.
\item SID blijft behouden als het object verplaatst wordt binnen hetzelfde domein. Als er verplaatst wordt naar een nieuw domein zal de SID wijzigen.
\item Wordt gegenereerd en toegewezen bij de aanmaak van het object.
\item sIDHistory, houdt alle SIDs bij die het SPN in het verleden had om te vermijden dat een gebruiker na verplaatsing van objecten zijn toegang zou verliezen.
\end{itemize}

\item \textbf{UPN - User Principal Name}
\begin{itemize}
\item Doel aanmeldingsnamen van gebruikers vereevoudigen.
\item opgeslagen in het userPrincipalName kenmerk.
\item Als de UPN enkel gebruikt wordt voor aanmelding, moet hij uniek zijn binnen het volledige forest.
\item Bestaat standaard uit [RDN gebruiker]@[UPN suffix] (zonder [ en ])
\item UPN suffix kan vervangen worden door
\begin{itemize}
\item DNS domeinnaam van het domein waar de account zich bevindt of het root domein
\item Mag zelfs een willekeurige naam zijn ook, als hij geregistreerd is met behulp van de Active Directory domeins and Trust snap-in.
\end{itemize}
\item Wordt maar sporadisch gebruikt door compabiliteitsredenen. Vaak maakt men gebruik van: [NetBIOSnaam van het domein]-[SAM accountnaam]. (zonder [ en ]).
\end{itemize}

\item \textbf{NetBIOS}
\begin{itemize}
\item Bestaat standaard uit de meest linkse component in de DNS naam van het domein
\item Is niet langer dan 15 letters
\item deze naam moet uniek zijn in zijn forest
\end{itemize}

\item \textbf{SAM accountnaam - Security Accounts Manager}
\begin{itemize}
\item Moet uniek zijn in het domein
\item Wordt opgeslagen in sAMAccountName
\item Bestaat uit hoogstens 20 karakters, standaard de eerste 20 bytes van de RDN afgesloten door een \$
\footnote{in de cursus staat er bytes p21, voorlaatste paragraaf, ik zou eerder denken dat het letters zijn}.
\item Deze naam kan op elk gewenst moment veranderd worden.
\end{itemize}
\clearpage
\item \textbf{DNS hostname}
\begin{itemize}
\item opgeslagen in dnsHostName kenmerk
\item standaard eerste 15 bytes van de RDN gevold door de suffix voor de primaire DNS
\item Standaard is de suffix de volledige DNS naam van het domein waar de computer toe behoord.
\item Er kan afgeweken worden; meer dan 15 chars en andere DNS naam.
\end{itemize}

\end{enumerate}

\section{Enkele veel gebruikte klassen vertonen nog \\ "meer identificerende attributen voor hun instanties.\\ Bespreek deze klassen en attributen.}

\textbf{Komt later aan bod. zaken zoals:}
\begin{enumerate}
\item lDAPDisplayName
\item Object identifier
\item \textbf{objectClass (de hi\"erarchische klassen)}
\item \textbf{objectCategory (de categorie van de klasse van het object)}
\item ... 
\end{enumerate}

\clearpage
\section{In welke partities is de Active Directory informatie\\ verdeeld ? Geef de betekenis van elke partitie,\\ hun onderlinge relatie,\\ en de replicatiekarakteristieken ervan. (laatste helft 2.2.3)}

\subsection{Wat?}
We noemen de verzameling van alle active directory informatie (objecten en containerobjecten samen met hun meta data (ook objecten)) het gegevensarchief of de directory. Elke domeincontroller bevat een kopie van de directory van zijn domein. De informatie is fysiek verdeeld in minimaal 3 categori\"en of partities. Cli\"ent computer houden (uiteraard) geen informatie bij.

\subsection{Partities}
\begin{enumerate}
\item \textbf{Domeinpartities met domeingegevens}
\begin{itemize}
\item bevatten informatie over objecten in het domein: gedeelde bronnen (servers, bestanden en printers) en accounts.
\item Bij installatie worden er een aantal standaard objecten aangemaakt, een daarvan is de administrator account
\item elk domein zit in een aparte partitie, er zijn dus evenveel partities met domeingegevens als dat er domeinen in het forest zijn.
\item deze gegevens hebben bijgevolg enkel betrekking op dit domein en worden niet gedistribueerd naar ander domeinen.
\item een subset van deze gegevens wordt opgeslagen in de global catalog
\end{itemize}

\item \textbf{Applicatie partities}
\begin{itemize}
\item bv dns gegevens
\item kunnen geen SPN objecten bevatten
\item kunnen niet verplaatst worden buiten de applicatie partitie
\item beschikbaar vanaf windows server 2003
\item zelf partities maken met adsiedit.msc
\end{itemize}

\clearpage

\item \textbf{configuratie gegevens}
\begin{itemize}
\item beschrijven de fysieke topologie van de directory (bv lijst van alle domeinstructuren, locaties van domeincontrollers en global catalog controllers, sites, ..)
\item Instellingen voor het hele forest worden vertaald naar kenmerken van objecten in de configuratie gegevens. (bv. uPNSuffixes kenmerk houdt de mogelijke UPN suffixen bij)
\end{itemize}

\item \textbf{schema}
\begin{itemize}
\item bevat een formele definitie van alle objecten en kenmerkgegevens die kunnen opgeslagen worden in de directory.
\item is uniek voor alle domeinen in het forest.
\end{itemize}
\end{enumerate}

\subsection{Onderlinge relatie}
\begin{itemize}
\item Logische structuur ; boomstructuur
\mijnfiguur[ht]{width=\textwidth}{structuur}{onderlinge relatie van de partities, uit de cursus "Besturingssystemen III - Windows Server (J. Moreau)"}
\item Het forest root domein staat bovenaan en bevat de domein partities samen met de configuratie partitie
\item partities kunnen deel uitmaken van een andere partitie, zo kan een domein partitie deel zijn van een hoger liggende domein partitie.
\item De schema partitie is een onderdeel van de configuratie partitie
\item Applicatie partities kunnen op 3 plaatsen toegevoegd worden
\begin{enumerate}
\item als een afzonderlijke boom in het forest
\item als kind van een domein partitie
\item als kind van een applicatie partitie
\end{enumerate}
\item Fysieke structuur: de schema partitie en de configuratie partitie zijn 2 verschillende entiteiten.
\end{itemize}

\subsection{Replicatie}
\begin{itemize}
\item Elke partitie is een aparte eenheid voor replicatie.
\item schema en configuratie gegeven worden gerepliceerd naar alle domeincontrollers in het forest. 
\item De domeingegevens van een bepaald domein worden gerepliceerd binnen het domein zelf.
\item de applicatie partities worden uitgewisseld met een eigen deelverzameling specifiek geconfigureerde domeincontrollers van het forest, onafhankelijk van de domein grenzen. (bv dns gegevens enkel syncen met dns servers)
\item een subset van de kenmerken van alle objecten in de domeingegevens van elk domein in het forest worden gerepliceerd naar de globale catalogus.
\end{itemize}
\chapter{attributeSchema objecten (2.2.4 en 2.2.5)}
\npar
Er zijn verschillende soorten schema's
\begin{itemize}
\item \textbf{Active directory schema - re\"ele schema} : volledige schema dat de regels van klassen en objecten bevat.
bevat 2 soorten definities:
\begin{itemize}
\item attributeSchema objecten: kenmerken, elk kenmerk wordt 1 keer gedefinieerd en wordt daarna gebruikt voor meerdere klassen.
\item classSchema objecten: klassen, de klassen die gemaakt kunnen worden.
\end{itemize}
\item \textbf{abstracte schema}: compacte representatie van het gehele schema
\end{itemize}

\section{Bespreek het doel en de werking van attributeSchema objecten. Hoe kunnen deze objecten het best geraadpleegd en gewijzigd worden ?}

\subsection{Doel \& werking}
\begin{itemize}
\item Kenmerken van klassen zijn zelf objecten in het schema
\item beperkingen opleggen
\item worden beheerd net als andere objecten
\item een kenmerk kan in meerdere klassen hergebruikt worden
\end{itemize}

\subsection{Raadplegen \& wijzigen}
\begin{itemize}
\item dsquery (tonen)
\item via adsiedit.msc
\item zelf gemaakte scripts
\item ldifde csvde
\item verwijderen van items is niet mogelijk, wel isDefunct op true zetten, zodat ze niet meer aangemaakt kunnen worden. Voordeel hiervan is dat ongedaan maken van een (foutieve verwijdering) eenvoudig is.
\end{itemize}

\section{Bespreek de diverse naamgevingen van attributeSchema objecten.}

Voor elk object is ook een viervoudige naamgeving aanwezig.
\begin{enumerate}
\item \textbf{CN - Common name}
\begin{itemize}
\item Niets anders dan de RDN van het attributeSchema object in de schema container.
\item bijgehouden in het cn attribuut
\end{itemize}

\item \textbf{GUID van een kenmerk}
\begin{itemize}
\item Onafhankelijk van het GUID van een attributeSchema object (duh)
\item automatisch gegenereerd ndien gewenst
\item uniek binnen het forest
\item bijgehouden in het schemaIDGUID attribuut
\end{itemize}

\item \textbf{LDAP display name}
\begin{itemize}
\item belangrijk voor programmatische toegang
\item bijgehouden in het lDAPDisplayName attribuut
\end{itemize}

\item \textbf{OID - object identifier}
\begin{itemize}
\item interne representatie
\item x.500 ids worden verleend door speciale autoriteiten zoals ITU ANSI en ISO en zijn gegarandeerd uniek in alle netwerken over de hele wereld.
\item je kan een tak aanvragen of een unieke genereren in de ms subtak met behulp van de oidgen
\item bijgehouden in het attributeID attribuut.
\end{itemize}
\end{enumerate}

\section{Bespreek de belangrijkste kenmerken van attributeSchema objecten, en hoe die ingesteld kunnen worden.}
De 7 belangrijke kenmerken zijn
\begin{enumerate}
\item \textbf{attributeSyntax \& oMSyntax}
\begin{itemize}
\item bepaald het data type (26 mogelijkheden waarvan 18 in gebruik, bv boolean, integer)
\item het is niet mogelijk om nieuwe syntax te defini\"eren.
\item de oMSyntax wordt gebruikt om een bijkomend onderscheid te maken omdat de attributeSyntax alleen niet genoeg blijkt te zijn.
\end{itemize}


\item \textbf{rangeLower en rangeUpper}: bereikbeperkingen van een kenmerken

\item \textbf{isSingleValued} : Of het object een over meerdere waarden heeft

\item \textbf{searchFlags} : binaire nformatie waarbij de meeste bits bepalen of het kenmerk op een of andere manier ge\"indexeerd wordt. Indien het kenmerk ge\"indexeerd is, kan er sneller gezocht worden op dat kenmerk.
\begin{itemize}
\item laagste bit: eenvoudige indexering van de waarde van het kenmerk
\item tweede laagste bit: waarde van het kenmerk combineren met de identificatie van de container. Dergelijke containerized indexen kunnen snel alle objecten binnen een specifieke container opsporen.
\item derde laagste bit: ambiguous name resolution toelaten. Zoeken waarbij minstens een kenmerk uit een verzameling kenmerken een specifieke waarde aanneemt.
\item zesde laagste bit; versnellen van opzoekingen waarin kenmerken met jokertekens vermeld worden. deze tuple indexen vergen heel wat resources en worden best in beperkte mate gebruikt.

\item vijfde laagste bit: heeft niets met indexing te maken, maar bepaald of de waarde van het attribuut behouden blijft indien men een kopie maakt van het object.
\end{itemize}

\item \textbf{systemFlags} Bevat ook binaire informatie
\begin{itemize}
\item de laagste bit bepaald of het kenmerk al dan niet gerepliceerd wordt naar andere domeincontrollers. Niet gerepliceerde kenmerken worden gebruikt voor caching of gebruikt bij relatief dynamische kenmerken waarvan de waarde grequent wijzigt zoals lastLogion en LAstLogoff.
\item het derde laagste bit van systemFlags wijst op een geconstrueerd attribuut; een attribuut dat telkens opnieuw berekend wordt.
\end{itemize}

\item \textbf{isMemberOfPartialAttributeSet}: bepaald of het kenmerk in de global catalog opgenomen wordt of niet.

\item \textbf{linkID}
\begin{itemize}
\item Sommige kenmerken vormen koppels bestaande uit forward-link en back-link kenmerken
\item De referenti\"ele integriteit te garanderen.
\item Enkel de forward link kan aangepast worden, de backlink wordt beheerd door het systeem.
\item gebruik door de overeenkomstige attributen van de kenmerken op te vullen met opeenvolgende even en oneven gehele getallen.
\end{itemize}
\end{enumerate}

\section{Welke andere types objecten bevat het Active Directory schema, en wat is hun bedoeling ? (o.a. 2.2.7)}

\begin{enumerate}
\item \textbf{classSchema-objecten} Groepen de attributen per klasse en geven dus aan welke klassen er gemaakt kunnen worden.
\item \textbf{het abstracte schema}
\begin{itemize}
\item abstracte schema
\item vereenvoudige interface aan LDAP cli\"ents door het verbergen van overbodige implementatie details.
\item RDN = Aggregate
\item high level toegang via ADSI interfaces.
\item geeft beperkt aantal kenmerken aan van het class- en attributeSchema.
\item kan heel wat werk besparen.
\end{itemize}
\end{enumerate}

\clearpage
\section{Via welke attributen kun je de klasse van een willekeurig Active Directory object achterhalen ? Hoe moet je op zoek gaan naar alle objecten van een bepaalde klasse ? Illustreer aan de hand van relevante voorbeelden. (laatste paragraaf 2.2.6)}

Hiervoor kan men gebruik maken van 2 mandatory-kenmerken van de top klasse. Deze zijn, doordat ze mandatory zijn in de topklasse, verplicht aanwezig in elk object.

\subsection{objectClass}
Is een multi valued en niet ge\"indexeerd attribuut dat alle hi\"erarchische superklassen (op de statische hulpklassen na) bevat.

\subsection{objectCategory}
Is een single valued en ge\"indexeerd attribuut dat de meest typische vertegenwoordiger uit de verzameling bestaande uit de klasse zelf en alle hi\"erarchische superklassen.

\subsection{Wat gebruiken?}
\begin{itemize}
\item Als de objectCategory is ingesteld met de klasse van het object is dit natuurlijk de beste keuze. De opzoeking laat toe om een beroep te doen op indexeren, wat een stuk performanter is.
\item objectCategory ingesteld op een hogere klasse: problemen: we krijgen ook andere deelklassen. Het beste is om eerst de hogere klasse op te halen en dan deze kleinere lijst nogmaals filteren. 
\item alleen de objectClass selecteren is het traagste vermits er niet ge\"indexeerd wordt.
\end{itemize}
































\chapter{classSchema objecten (2.2.4 en 2.2.6)}
\section{Bespreek het doel en de werking van classSchema objecten.}
\begin{enumerate}
\item \textbf{doel \& werking}
\begin{itemize}
\item doel: defini\"eren hoe een klasse er uit ziet.
\item voor elke klasse is er een object in het classSchema
\item 2 soorten regels
\begin{enumerate}
\item structuurregels: hi\"erarchische relaties tussen de klassen
\item inhoudsregels: kenmerken defini\"eren die de klasse moet hebben (verwijzen naar attributen van hierboven)
\end{enumerate}
\item worden ook opgeslagen als objecten.
\end{itemize}
\end{enumerate}

\section{Hoe benadert Active Directory het mechanisme van overerving ?}

\begin{enumerate}
\item \textbf{basis overerving}
\begin{itemize}
\item elke klasse, behalve de TOP klasse is afgeleid van een andere klasse
\item superklasse of bovenliggende klasse
\item subklassen 
\item overerving: overnemen van alle kenmerken; structuur- en inhoudsregels van de bovenliggende klasse. (!! kenmerken GUID overnemen, maar daarom wordt de waarde van GUID nog niet overgenomen)
\item overname is recursief, alle kenmerken van alle bovenliggende klassen overnemen
\end{itemize}

\item \textbf{meervoudige overerving}
\begin{itemize}
\item enkel kenmerken overnemen van onmiddellijke superklasse en speciaal bestemde hulpklassen.
Deze hulpklassen kunnen zelf geen klassen genereren.
\item hulpklassen kunnen zowel dynamisch als statisch gebruikt worden
\end{itemize}

\item \textbf{statisch gebruik van hulpklassen}
\begin{itemize}
\item auxilairyClass of systemAuxiliaryClass
\item Wordt vastgelegd in de definitie van de klasse
\item statisch en kan niet veranderd worden zonder de definitie van de klasse aan te passen.
\item in de definitie van de klasse wordt reeds vastgelegd van welke hulpklassen deze klasse kenmerken kan overnemen. Alle objecten hebben dit kenmerk dus vanzelf.
\end{itemize}

\item \textbf{dynamisch gebruik van hulpklassen}
\begin{itemize}
\item beschikbaar vanaf windows server 2003
\item entryTTL als optioneel attribuut dat gebruikt kan worden om de klasse vanzelf te laten vervallen
\item entryTTL kan opgefrist worden
\item dynamisch kenmerken bijladen door het objectClass kenmerken aan te vullen met de klassenaam. Hierdoor erft enkel deze instantie de kenmerken van de hulpklasse
\item dit moet wel bij de create van de instantie ingesteld worden vermits het objectClass kenmerk na creatie niet meer gewijzigd kan worden.
\end{itemize}

\end{enumerate}
Of een klasse een structureel, abstracte of hulpklasse is, is af te leiden uit de objectClassCategory integer.
de waarde van de ints zijn respectievelijk 1, 0 of 2 , en 3. Abstracte klasse kunnen ook geen objecten maken en worden gebruikt om objecten te verzamelen bv aPersoon . een abstracte klasse kan enkel een abstracte klasse als superklasse hebben.

\clearpage

\section{Bespreek de diverse naamgevingen van classSchema objecten.}

Er is voor elke object ook een viervoudige naamgeving:
\begin{enumerate}
\item common name
\item GUID
\item ldap displayname
\item object ID
\end{enumerate}
Voor meer informatie kijk bij reeks1 vraag2, daar worden de verschillende namen al uitgelegd.

\section{Bespreek de belangrijkste kenmerken van classSchema objecten, en hoe die ingesteld kunnen worden.}

2 soorten regels. Als de naam begint met system wordt het kenmerk beheerd door het systeem zelf.
\begin{enumerate}
\item \textbf{Inhoudsregels}: defini\"eren welke kenmerken een klasse heeft.
\begin{itemize}
\item \textbf{postSuperiors \& systemPostSuperiors} lijst van verplichte kenmerken.Deze zullen na overerving ook altijd verplicht blijven ook al zijn ze in de klasse zelf als optioneel gemarkeerd.
\item \textbf{mayContain \& systemMayContain} lijst van optionele kenmerken
\item \textbf{rDNAttID} bepaald welk kenmerk van een klasse gebruikt wordt om de RDN van objecten te bepalen. Voor de meeste klassen staat dit kenmerk ingesteld op de waarde.
\item \textbf{defaultSecurityDescriptor} bepaalt expliciete machtigingen die gelden voor objecten van deze klassen. Dit kan gebruikt worden om het beheer van deze klasse te delegeren.
\item \textbf{systemOnly} indien dit kenmerk de waarde true heeft kunnen de structuurregels en inhoudsregels niet gewijzigd worden.
\item isDefunct kan gebruik worden om klassen te deactiveren (verwijderen van klassen is niet mogelijk). Doordat klassen enkel gedeactiveerd kunnen worden is het ongedaan maken ervan veel simpeler.
\end{itemize}
\clearpage
\item \textbf{Structuur regels} : Defini\"eren de mogelijke hi\"erarchische relaties tussen klassen of objecten.
\begin{itemize}
\item \textbf{objectClassCategory} : categorie van de klasse bepalen. 
\begin{itemize}
\item 1 : structurele klasse
\item 0 of 2 : abstracte klasse
\item 3 hulpklasse
\end{itemize}
\item \textbf{objectClass} multivalued niet ge\"indexeerd. Alle bovenliggende klassen
\item \textbf{objectCategory}: single valued en ge\"indexeerd. De meest typische vertegenwoordiger.
\item \textbf{auxiliaryClass en systemAuxiliaryClass}  welke hulpklassen de klasse heeft
\item \textbf{poosSuperiors en systemPossSuperiors} bepaald welke andere objecten de klasse kunnen bevatten. Bijvoorbeeld een organizationalUnit kan bijvoorbeeld container zijn voor o.a. user-obecten, dit betekent dat het classSchema-object voor user verwijst naar organizationalUnit.
\end{itemize}
\end{enumerate}

\section{Welke andere types objecten bevat het Active Directory schema, en wat is hun bedoeling ? (o.a. 2.2.7)}

Naast het classSchema-objecten zijn er ngo 2 andere beschikbaar:
\begin{enumerate}
\item \textbf{attributeSchema-objecten} waar de attributen gedefini\"eeerd worden. zie hierboven ergens.
\item \textbf{abstract schema objecten} een compacte representatie van het re\"eele schema, zie hierboven.
\end{enumerate}
\clearpage
\section{Hoe en met welke middelen kan het Active Directory schema uitgebreid worden ? (2.2.8, ldifde fractie 2.2.3)}

\textbf{risicovol}
\begin{itemize}
\item risico vol
\item schema uitbreiden = wijzigingen geldig in heel het forest
\item maak zoveel mogelijk gebruik van overerving om problemen te vermijden en maak dus enkel een geheel nieuwe structurele klasse aan als er geen enkel bestaande klasse voldoet.
\item veel potentieel / veel mogelijkheden
\item uiteraard kunnen enkel gemachtigde gebruikers wijzigingen doorvoeren
\item best een testomgeving gebruiken
\end{itemize}

textbf{tools}
\begin{enumerate}

\item \textbf{ldifde en csvde}
\begin{itemize}
\item via command prompt
\item grootschalige wijzigingen
\item uitwisseling gebaseerd op intermediaire tekstbestanden in:
\begin{itemize}
\item ldifde= LDAP Data Interchange Format
\item csvde = comman seperated value
\end{itemize}
\end{itemize}


\item \textbf{LDAP of ADSI interface} Wijzigingen doorvoerenvia ADSIEdit.msc of eigen scripts.

\item \textbf{Active Directory schema snap-in}
\begin{itemize}
\item standaard niet in MMC (Microsoft MAnagement console).
\item klassen en attributen in verschillende mappen weergegeven
\item na het selecteren van een klasse in het linkerpanel krijg je een detail overzicht
\item dubbelklikken op een kenmerk laat toe om wijzigingen aan te brengen via een dialoog venster.
\end{itemize}


\end{enumerate}















\chapter{Active Directory functionele niveaus (2.4.3)}
\section{Geef de diverse functionele niveaus waarop Active Directory kan ingesteld worden, en welke beperkingen er het gevolg van zijn.}

\subsection{Domein functioneel niveau}
Het domein functioneel niveau geeft aan welke minimum eis er gesteld wordt aan het besturingssysteem van de domeincontrollers en bepaalt tegelijkertijd welk faciliteiten er beschikbaar zijn. Dit niveau wordt opgeslagen in 2 attributen = ntMixedDomain en msDS-Behavior-Version. Er zijn vier mogelijkheden
\begin{enumerate}
\item \textbf{Windows 2000 mixed}
\begin{itemize}
\item de laagste active directory functionaliteit
\item windows 2008+ domeincontrollers niet mogelijk
\item standaard gebruikt bij windows server 2000 en 2003
\end{itemize}

\item \textbf{Windows 2000 native} enkel windows server NT5+  domein controllers, werkposten en lidservers mogen lager zijn
\item \textbf{Windows Server 2003} enkel windows server 2003+ domein controllers, werkposten en lidservers mogen lager zijn.
\item \textbf{Windows Server 2008} enkel windows server 2008+ domein controllers, werkposten en lidservers mogen lager zijn.
\end{enumerate}

\subsection{Forest functioneel niveau}
Analoog aan het domein funcitoneel niveau is er ook een forest functioneel niveau. Dit functioneel niveau bepaald het minimale niveau van een alle domeinen binnen een forest. Dit wordt opgeslagen in het msDS-Behavior-Version attribuut, maar dan van het partitions containerobect van de configuratie gegevens. Er zijn 3 mogelijkheden:
\begin{enumerate}
\item \textbf{windows 2000 forest} geen enkele eis aan het functioneel niveau van de liddomeinen. De standaard instelling

\item \textbf{Windows Server 2003} enkel domeinen van 2003+ niveaus 
\item \textbf{Windows Server 2008} enkel domeinen van 2008+ 
\end{enumerate}



\section{Bespreek van elk niveau alle eraan gekoppelde voordelen. Geef hierbij telkens een korte bespreking (verspreid over de cursus !) van de ingevoerde begrippen.}

\subsection{Domein functioneel niveau}
\subsubsection{Windows 2000 native}
\begin{itemize}
\item 1 globale catalog voor het ganse forest
\item transitieve vertrouwensrelaties tussen verschillende domeinen van eenzelfde forest
\item alle domeincontrollers kunnen zelfstandig een aantal SPN objecten aanmaken door de delegering van de RID master, waar er bij het mixed domein steeds beroep moet doen op de PDC emulator master van een domein.
\item ruimere mogelijkheden voor configuratie van groepen
\item alle sids die een SPN object in het verleden gehad heeft, worden bijgehouden in het sIDHistory kenmerk.
\end{itemize}
\clearpage
\subsubsection{Windows Server 2003 niveau}
\begin{itemize}
\item gebruik van aanvullende schema klassen en attributen 
\item naam van een domeincontroller kan eenvoudiger veranderd worden (geen degradatie en promotie meer nodig)
\item dagvullende opdrachten: rdirusr en redircmp om de default active directories te wijzigen waarin respectievelijk nieuwe gebruikers en nieuwe computers terecht komen.
\item caching op domeincontroller niveau van UPN suffices en het lidmaatschap van universele groepen zodat het niet meer strikt noodzakelijk is dat tijdens het inlogproces een global catalog bereikbaar is
\item group policies filteren ook met behulp van WMI scripts.
\end{itemize}

\subsubsection{Windows Server 2008}
\begin{itemize}
\item aanvullende schema klassen en attributen 
\item encryptie van het kerberos protocol
\item fijnkorrelig wachtwoord beleid
\item replicatie van DFS namesspaces en SYSVOL share via DFS replicatie (performanter van file replication service)
\end{itemize}

\subsection{Forest functioneel niveau}
\subsubsection{Windows 2000}
\begin{itemize}
\item geen enkele eis aan het functionele niveau van de liddomeinen
\end{itemize}
\clearpage
\subsubsection{Windows Server 2003}
\begin{itemize}
\item hergebruiken van gedeactiveerde attributen en klassen
\item dynamische hulpklassen
\item dynamische objecten met een beperkte levensduus
\item effici\"entere replicatie van de global catalog gegevens, waardoor ondermeer toevoeging van een nieuw kenmerk niet leidt tot een volledige synchronisatie van alle objectkenmerken.
\item veranderen van naamgeving en de hi\"erarchische structuur van domeinen in een forest.
\item transitieve vertrouwens relaties tussen verschillende forests
\item read only windows server 2008+ domeincontrollers
\item effici\"entere KCC algoritmen voor de constructie van de replicatietopologie.
\item replicatie van individuele waarden van multi-valued attributen zodat bijvoorbeeld bij verandering van het lidmaatschap van een groep niet de volledige verzameling leden opnieuw moet gesynced worden.
\end{itemize}

\subsubsection{Windows Server 2008}
\begin{itemize}
\item geen aanvullende functionaliteiten
\item wel betere beveiliging
\end{itemize}

\section{Hoe kan men detecteren op welk niveau een Active Directory omgeving zich bevindt ?}
\subsection{Domein functioneel niveau}
\begin{itemize}
\item opgeslagen in 2 attributen
\begin{enumerate}
\item ntMixedDomain
\item msDS-Behavior-Version
\end{enumerate}
\item ingesteld op het domeinobject zelf (bv. dc=iii, dc= hogent , dc = be)
\end{itemize}

\subsection{Forest functioneel niveau}
\begin{itemize}
\item ingesteld op het partitions containerobject van de configuratiegegevens
\item wordt bepaald door 1 attribuut: msDS-Behavior-Version
\end{itemize}

\section{Op welke diverse manieren kan men het functionele niveau verhogen of verlagen ?}

\begin{itemize}
\item gebeurd niet automatisch, moet manueel gebeuren
\item Kan op 2 manieren
\begin{enumerate}
\item zelf de attributen manipuleren
\item de Active Directory Domains and Trust snap-in
\end{enumerate}
\item Als niet alle voorwaarden voldaan zijn wordt je hiervan op de hoogte gebracht
\item verlagen is niet mogelijk
\item alle domeincontrollers moeten opnieuw opgestart worden om de wijzigingen door te voeren.
\item als je weet dat er geen oudere controllers zijn kun je best het niveau al updaten bij de installatie van de eerste active directory controller zodat alle controllers die erna toegevoegd worden vanzelf het juiste niveau hebben.
\end{itemize}
\chapter{Active Directory domeinstructuren (2.4.4 en 2.4.6)}
\section{Wat is de bedoeling van vertrouwensrelaties ?}
\begin{itemize}
\item Gebruikers van 1 trusted domein ook vertrouwen/ kunnen verifi\"eren in een ander trusting domein.
\item weergegeven et een pijl in de richting van het trusted domein. (Als domein A domein B vertrouwd, is er een pijl van A naar B)
\item eenmaal geverifieerd moet er gekeken worden naar de rechten en machtigingen van een gebruiker alvorens hij toegang krijgt tot het andere domein. Deze machtigingen worden per domein toegewezen.
\end{itemize}

\section{Bespreek de verschillende soorten vertrouwensrelaties.}
\subsection{expliciet}
Deze moeten manueel aangemaakt worden
\subsubsection{forest trusts}
\begin{itemize}
\item windows server 2003+ (2003 of hoger) nodig
\item manueel tussen de rootdomeinen van de forests
\item directionele
\item transitief
\item realms bestaande uit meer dan 2 forests? voor elk koppel een trust maken
\end{itemize}

\subsubsection{Realm trusts}
\begin{itemize}
\item veralgemening van forest trust
\item tussen windows server 2008+ en willekeurige kerberos v5 realms onafhankelijk van het besturingssysteem waarop die ge\"implementeerd zijn.
\item bidirectioneel / enkelvoudig 
\item transitief / niet-transitief
\end{itemize}
 
\subsubsection{Verkorte vertrouwensrelaties}
\begin{itemize}
\item worden gebruikt om het vertrouwenspad in grote en complexe trees korter te maken.
\item performanter
\item ook shortcut of cross-link genoemd
\item in praktijk pas zinvol als het vertrouwens pad 5 of meer domeinen overspant.
\item enkelvoudig/bi-directioneel
\end{itemize}
 
\subsubsection{Externe vertrouwensrelaties} 
\begin{itemize}
\item een domein vertrouwd het andere
\item niet transitief
\item altijd enkelvoudig, wil je bi-directioneel dan moet je 2 relaties maken.
\item kan niet binnen hetzelfde forest
\item bedoeld voor communicatie met externe partners
\item met oude NT4 domeinen
\end{itemize}

\subsection{impliciet}
Deze worden automatisch aangemaakt en beheerd bv tussen rootdomein en de subdomeinen

\clearpage
\section{Op welke diverse manieren kunnen vertrouwensrelaties gecre\"eerd en gecontroleerd worden ? Bespreek ook de optionele configuratiemogelijkheden.}
Enkel de expliciete vertrouwensrelaties moeten zelf geconfigureerd worden.

\begin{enumerate}
\item \textbf{Active Directory and Domains Trust snap)in}
\begin{itemize}
\item beschikbaar via domain.msc
\item rechtermuisknop op domein, properties, trusts-tab, new trusts wizard.
\item aan elke vertrouwensrelatie wordt een wachtwoord toegewezen
\item aanvullende config is mogelijk en aangeraden vanuit het beveiligingsstandpunt
\begin{enumerate}
\item Standaard worden alle gebruikers van het trusted domein opgenomen in de authenticated users impliciete groep van het trusting domein.
\item men kan echter ook voor selective authentication kiezen waardoor dit per individuele gebruiker of gebruikersgroep expliciet moet ingesteld worden.
\item inden men gebruik maakt van SID Filtering (de standaard instelling), dan wordt enkel rekening gehouden met de SID opgeslagen in het objectSid attribuut van de objecten in het trusted domein (en bijgevolg met SIDs waarvan de domain subauthority identifier zeker overeenkomt met die van het trusted domein.
\item indein men SID Filtering uitschakeld, dan verwerkt het trusting domein ook de SIDs opgeslagen in het siDHistory attribuut. Malafide beheerder in het trusted domein met olledige toegang tot het siDHistory attribuut van de objecten in hun eigen domein kunnen langs deze weg zichzelf meer machtigingen en rechten toe-eigenen in het trusting domein.
\end{enumerate}
\end{itemize}

\item \textbf{Via command line}
\begin{itemize}
\item netdom trust : nieuwe relaties toevoegen
\item netdom query trust : huidige vertrouwensrelaties opvragen / query'n
\end{itemize}
\end{enumerate}

\clearpage
\section{Welke verschillen zijn er in praktijk tussen NT 4.0 en Windows Server domeinstructuren ? Bespreek de alternatieve mogelijkheden bij de conversie van een NT 4.0 domeinstructuur naar een Windows Server omgeving.}
\begin{itemize}
\item NT4 maakt een onderscheid in master domeinen en resource domeinen
\begin{itemize}
\item Master domein of account domein bevat de gebruikers en groepen 
\item resource domein biedt bronnen aan zoals printers, shared , ...
\item NT4 domeinstructuren bestaan uit een of meerder master domeinen en meerder resource domeinen. Er wordt een bi-directionele vertrouwensrelatie aangemaakt tussen all masterdomeinen onderling. Daarnaast zijn er enkelvoudige vertrouwensrelaties waarbij elk resource domein elk masterdomein vertrouwd.
\end{itemize}
\item omschakeling moet geleidelijk en evolutionair gebeuren ipv revolutionair.

\item windows server kan ervoor zorgen dat het aantal domeinen vermindert, wat het beheer zal vereenvoudigen
\begin{itemize}
\item gebruik maken van ou om domeinen te vervangen
\item oud hebben als bijkomend voordeel dat het verplaatsen van objecten veel makkelijker is.
\end{itemize}

\item beginnen in de root en dan naar beneden werken. eerst het master NT4 domein en dan de resource domeinen. Dit moet zo gebeuren omdat windows server altijd een root domein nodig heeft om te kunnen functioneren.

\item indien bedrijfseenheden als aparte organisaties moeten behandeld worden is een forest met aparte trees een zeer goee oplossing. Elke groep beheerders kan dan een eigen beveiligingsbeleid instellen dat onafhankelijk is van het beleid in andere domeinen. Daarbij worden gebruikers wel best verplaatst naar de domeinen met de bronnen die ze het meeste gebruiken.

\clearpage
\item indien er meerdere master domeinen waren , dan was dat om een van volgende redenen:
\begin{itemize}
\item \textbf{het netwerk is te groot , een SAM database groter dan 40MB is niet stabiel}
Dit is opgelost in windows server, hier neemt een database van 1000 000 gebruikers 20GB in, wat moeiteloos ondersteun kan worden met de huidige database technologie\"en. 
	
\item \textbf{het netwerk heeft verschillende geografische locaties}, aan elkaar gekoppeld door trage verbindingen waarover geen massaal replicatie verkeer tussen domeincontrollers gewenst is. Ook opgelost, zie hierboven.

\item \textbf{men wil een specifiek wachtwoord beleid voor verschillende groepen gebruikers} Dit kan opgelost worden door de fine-grained password policies van Windows Server 2008.

\item \textbf{het domeinmodel weerspiegeld de organisatie} waarin verschillende bedrijfseenheden controle moeten hebben over hun eigen gebruikers en bronnen. Dit is het enige argument om de aparte domeinen in de verschillende sites te behouden. Hierbij krijg je een root domein dat enkel een structural of placeholder domain is.
\end{itemize}
\end{itemize}

\end{document}