\chapter{classSchema objecten (2.2.4 en 2.2.6)}
\section{Bespreek het doel en de werking van classSchema objecten.}
\begin{enumerate}
\item \textbf{doel \& werking}
\begin{itemize}
\item doel: defini\"eren hoe een klasse er uit ziet.
\item voor elke klasse is er een object in het classSchema
\item 2 soorten regels
\begin{enumerate}
\item structuurregels: hi\"erarchische relaties tussen de klassen
\item inhoudsregels: kenmerken defini\"eren die de klasse moet hebben (verwijzen naar attributen van hierboven)
\end{enumerate}
\item worden ook opgeslagen als objecten.
\end{itemize}
\end{enumerate}

\section{Hoe benadert Active Directory het mechanisme van overerving ?}

\begin{enumerate}
\item \textbf{basis overerving}
\begin{itemize}
\item elke klasse, behalve de TOP klasse is afgeleid van een andere klasse
\item superklasse of bovenliggende klasse
\item subklassen 
\item overerving: overnemen van alle kenmerken; structuur- en inhoudsregels van de bovenliggende klasse. (!! kenmerken GUID overnemen, maar daarom wordt de waarde van GUID nog niet overgenomen)
\item overname is recursief, alle kenmerken van alle bovenliggende klassen overnemen
\end{itemize}

\item \textbf{meervoudige overerving}
\begin{itemize}
\item enkel kenmerken overnemen van onmiddellijke superklasse en speciaal bestemde hulpklassen.
Deze hulpklassen kunnen zelf geen klassen genereren.
\item hulpklassen kunnen zowel dynamisch als statisch gebruikt worden
\end{itemize}

\item \textbf{statisch gebruik van hulpklassen}
\begin{itemize}
\item auxilairyClass of systemAuxiliaryClass
\item Wordt vastgelegd in de definitie van de klasse
\item statisch en kan niet veranderd worden zonder de definitie van de klasse aan te passen.
\item in de definitie van de klasse wordt reeds vastgelegd van welke hulpklassen deze klasse kenmerken kan overnemen. Alle objecten hebben dit kenmerk dus vanzelf.
\end{itemize}

\item \textbf{dynamisch gebruik van hulpklassen}
\begin{itemize}
\item beschikbaar vanaf windows server 2003
\item entryTTL als optioneel attribuut dat gebruikt kan worden om de klasse vanzelf te laten vervallen
\item entryTTL kan opgefrist worden
\item dynamisch kenmerken bijladen door het objectClass kenmerken aan te vullen met de klassenaam. Hierdoor erft enkel deze instantie de kenmerken van de hulpklasse
\item dit moet wel bij de create van de instantie ingesteld worden vermits het objectClass kenmerk na creatie niet meer gewijzigd kan worden.
\end{itemize}

\end{enumerate}
Of een klasse een structureel, abstracte of hulpklasse is, is af te leiden uit de objectClassCategory integer.
de waarde van de ints zijn respectievelijk 1, 0 of 2 , en 3. Abstracte klasse kunnen ook geen objecten maken en worden gebruikt om objecten te verzamelen bv aPersoon . een abstracte klasse kan enkel een abstracte klasse als superklasse hebben.

\clearpage

\section{Bespreek de diverse naamgevingen van classSchema objecten.}

Er is voor elke object ook een viervoudige naamgeving:
\begin{enumerate}
\item common name
\item GUID
\item ldap displayname
\item object ID
\end{enumerate}
Voor meer informatie kijk bij reeks1 vraag2, daar worden de verschillende namen al uitgelegd.

\section{Bespreek de belangrijkste kenmerken van classSchema objecten, en hoe die ingesteld kunnen worden.}

2 soorten regels. Als de naam begint met system wordt het kenmerk beheerd door het systeem zelf.
\begin{enumerate}
\item \textbf{Inhoudsregels}: defini\"eren welke kenmerken een klasse heeft.
\begin{itemize}
\item \textbf{postSuperiors \& systemPostSuperiors} lijst van verplichte kenmerken.Deze zullen na overerving ook altijd verplicht blijven ook al zijn ze in de klasse zelf als optioneel gemarkeerd.
\item \textbf{mayContain \& systemMayContain} lijst van optionele kenmerken
\item \textbf{rDNAttID} bepaald welk kenmerk van een klasse gebruikt wordt om de RDN van objecten te bepalen. Voor de meeste klassen staat dit kenmerk ingesteld op de waarde.
\item \textbf{defaultSecurityDescriptor} bepaalt expliciete machtigingen die gelden voor objecten van deze klassen. Dit kan gebruikt worden om het beheer van deze klasse te delegeren.
\item \textbf{systemOnly} indien dit kenmerk de waarde true heeft kunnen de structuurregels en inhoudsregels niet gewijzigd worden.
\item isDefunct kan gebruik worden om klassen te deactiveren (verwijderen van klassen is niet mogelijk). Doordat klassen enkel gedeactiveerd kunnen worden is het ongedaan maken ervan veel simpeler.
\end{itemize}
\clearpage
\item \textbf{Structuur regels} : Defini\"eren de mogelijke hi\"erarchische relaties tussen klassen of objecten.
\begin{itemize}
\item \textbf{objectClassCategory} : categorie van de klasse bepalen. 
\begin{itemize}
\item 1 : structurele klasse
\item 0 of 2 : abstracte klasse
\item 3 hulpklasse
\end{itemize}
\item \textbf{objectClass} multivalued niet ge\"indexeerd. Alle bovenliggende klassen
\item \textbf{objectCategory}: single valued en ge\"indexeerd. De meest typische vertegenwoordiger.
\item \textbf{auxiliaryClass en systemAuxiliaryClass}  welke hulpklassen de klasse heeft
\item \textbf{poosSuperiors en systemPossSuperiors} bepaald welke andere objecten de klasse kunnen bevatten. Bijvoorbeeld een organizationalUnit kan bijvoorbeeld container zijn voor o.a. user-obecten, dit betekent dat het classSchema-object voor user verwijst naar organizationalUnit.
\end{itemize}
\end{enumerate}

\section{Welke andere types objecten bevat het Active Directory schema, en wat is hun bedoeling ? (o.a. 2.2.7)}

Naast het classSchema-objecten zijn er ngo 2 andere beschikbaar:
\begin{enumerate}
\item \textbf{attributeSchema-objecten} waar de attributen gedefini\"eeerd worden. zie hierboven ergens.
\item \textbf{abstract schema objecten} een compacte representatie van het re\"eele schema, zie hierboven.
\end{enumerate}
\clearpage
\section{Hoe en met welke middelen kan het Active Directory schema uitgebreid worden ? (2.2.8, ldifde fractie 2.2.3)}

\textbf{risicovol}
\begin{itemize}
\item risico vol
\item schema uitbreiden = wijzigingen geldig in heel het forest
\item maak zoveel mogelijk gebruik van overerving om problemen te vermijden en maak dus enkel een geheel nieuwe structurele klasse aan als er geen enkel bestaande klasse voldoet.
\item veel potentieel / veel mogelijkheden
\item uiteraard kunnen enkel gemachtigde gebruikers wijzigingen doorvoeren
\item best een testomgeving gebruiken
\end{itemize}

textbf{tools}
\begin{enumerate}

\item \textbf{ldifde en csvde}
\begin{itemize}
\item via command prompt
\item grootschalige wijzigingen
\item uitwisseling gebaseerd op intermediaire tekstbestanden in:
\begin{itemize}
\item ldifde= LDAP Data Interchange Format
\item csvde = comman seperated value
\end{itemize}
\end{itemize}


\item \textbf{LDAP of ADSI interface} Wijzigingen doorvoerenvia ADSIEdit.msc of eigen scripts.

\item \textbf{Active Directory schema snap-in}
\begin{itemize}
\item standaard niet in MMC (Microsoft MAnagement console).
\item klassen en attributen in verschillende mappen weergegeven
\item na het selecteren van een klasse in het linkerpanel krijg je een detail overzicht
\item dubbelklikken op een kenmerk laat toe om wijzigingen aan te brengen via een dialoog venster.
\end{itemize}


\end{enumerate}














