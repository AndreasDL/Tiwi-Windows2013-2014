\chapter{Active Directory server rollen (2.4.7, 2.3 en fractie 2.4.2)}
Welke vragen moet men zich stellen na de initiële installatie van een Windows Server toestel, in verband met bijzondere functies die de server kan vervullen met betrekking tot Active Directory ? Formuleer bij het beantwoorden van deze vragen telkens (voor zover relevant):
\section{Hoe bepaald wordt welke servers een dergelijke specifieke functie vervullen ? Hoeveel zijn er nodig (in termen van: minimaal/exact/maximaal aantal, in functie van ...), en waarom ?}
\section{Eigenschappen zoals bedoeling, noodzaak, kriticiteit, inhoud, synchronisatie, voor welke Windows versie(s) van toepassing, ... ?}
\section{De eventuele relatie tussen de diverse functies. Vermeld bijvoorbeeld welke functies al dan niet door dezelfde server kunnen vervuld worden, of misschien wel juist wel door dezelfde server moeten vervuld worden.}
\section{Op welke diverse manieren men de toewijzing van elke bijzondere functie kan instellen, wijzigen en/of ongedaan maken ?}

\clearpage
\textbf{Antwoord:}
Telkens men een nieuw Windows Server toestel aan het netwerk toevoegt, moet men zich na de initi\"ele installatie ervan enkele cruciale vragen stellen, in het bijzonder in verband met de rol die de server zal vervullen met betrekking tot Active Directory.

\begin{enumerate}
\item \textbf{Wordt de server al dan niet opgenomen in het domein?}
\begin{itemize}
\item een computer waar windows server op draait, maar die geen lid is van een werkgroep of domein wordt een zelfstandige server genoemd. Deze servers kunnen wel bronnen delen binnen het netwerk, maar profiteren niet van de vele voordelen die active directory biedt.
\item Om toch te kunnen genieten van deze voordelen besluit men meestal om deze server lid te maken van het domein, hij wordt dan een lidserver genoemd.
\end{itemize}

\item \textbf{Vervult de in een domein opgenomen server al dan niet de functie van domeincontroller?} 
\begin{itemize}
\item Indien hij de rol van domeincontroller niet vervult, wordt hij een lid- / memberserver genoemd.
\item een lidserver is niet betrokken bij de replicatie en zal bijgevolg dus ook geen zaken zoals inloggen, beleidinfo of beveiliging voorzien en/of afhandelen.
\item meestal zijn die file servers, toepassingsservers, database servers, webservers, firewall, routers .. Deze functies worden gegroepeerd in 3 niveaus:
\begin{enumerate}
\item \textbf{server rollen}: DHCP servers, dns servers, network policy and access services en web servers implementeren primaire serverfuncties.
\item \textbf{role services}: complexere servere rollen die optionele componenten bieden. Voorbeelden hiervan zijn windows en unix wachtwoorden synchronisatie. 
\item \textbf{Features}: Group policy management , powershell , windows server backup features en snmp services die voor meer ondersteunende functies zorgen.
\end{enumerate}
\item Dit is configureerbaar via de Add Roles , Add Role Services en Add Features wizards van de Server Manager.
\item lidservers blijven een eigen lokale beveiligingsdatabank  de Security Account Manager (SAM) behouden. zowel gebruikers die in de AD gedefinieerd zijn als gebruikers die louter in de lokale SAM bestaan , kunnen, gefilterd door het mechanisme van machtigingen en gebruikersrechten , gebruik maken van de faciliteiten van een lidserver.
\item men moet er rekening mee houden dat de promotie tot domein controller een aanzienlijke belasting met zich meebrengt.
\item ten minste 1 domeincontroller in elke site om performatie te verbeteren
\end{itemize}

\item \textbf{Als er gekozen wordt voor een domeincontroller, moet ook de functie van globale catalogus ondersteund worden?}
\begin{itemize}
\item om replicatie verkeer te beperken wordt er best een globale catalogus per site ingesteld.
\item veel controllers = veel replicatie verkeerd vs. geen controller in de site: zeer traag.
\item Indien er slechts een domein in het forest is, is er geen enkel bezwaar om van alle domeincontrollers een globale catalogus te maken.
\end{itemize}

\item \textbf{Welke domeincontrollers vervullen deze operations master rollen}
\begin{itemize}
\item \textbf{alle domeincontrollers van een domein zijn nagenoeg equivalent} Een aatal specifieke AD functies, operations master rollen genoemd, kunnen echter slechts door een enkele controller in het domein of het forest vervuld worden. 

\item \textbf{OM rollen Uniek in elk domein}
\begin{enumerate}
\item \textbf{RID master}
\begin{itemize}
\item een dc maakt SPN objecten aan en gebruikt daarvoor SIDs die hij uit een SIDS pool haalt.
\item indien de SID pool voor 80\% gebruikt is zal de dc een nieuwe SID pool aanvragen bij de RID master.
\item De pool bestaat uit 512 RIDs - Relative IDs.
\end{itemize}

\item \textbf{PDC emulator master}
\begin{itemize}
\item volledige emulatie van een NT4 primaire controller in een windows 2000 mixed domain met NT4 backup dc. Volledig transparant voor nt4 gebruikers en users.
\item enkel relevant in een windows 2000 mixed domain met NT4 backup DC's.
\item PDC emulator master krijgt voorgang bij de replicatie van wachtwoord wijzigingen. Als een wachtwoord veranderd wordt, is dit niet meteen op elke dc aanwezig, daarom zal een client bij een afwijzing ook de pdc contacteren vermits deze direct gesynchroniseerd zou moeten zijn.
\item is ook primaire bron voor tijdssynchronisatie.
\item best de RID en PDC door dezelfde controller laten vervullen.
\end{itemize}

\clearpage
\item \textbf{Infrastructure master}
\begin{itemize}
\item Verantwoordelijk voor het bijwerken van verwijzingen vanuit objecten in het eigen domein naar objecten in andere domeinen.
\item zie ook forward link, back link, phantom objects
\item indien een forward link wijst naar een object in een ander domein, kan het backlink kenmerk van de object niet rechtstreeks aangepast worden. AD lost dit op door in beide domeinen een phantom object te cre\"eren dat doorverwijst naar de DN, GUID en SID van de respectievelijke objecten in de andere domeinen. De infrastructure master van een domein vergelijk continu de kenmerken van zijn phantom objecten met de kenmerken van de corresponderende objecten in externe domeinen, en de kenmerken an phantom objecten in externe domeinen die doorverwijzen naar eigen objecten met de kenmerken van de objecten. (kijkt dus naar zijn eigen phantoms vs ander domein en vice versa). 
\item De infrastructure master zal de gegevens uiteindelijk bewerken door de global catalogus te contacteren.
\item dc die ook global catalog server zijn beschikken al over een kopie van de objecten van andere domeinen en zullen dus geen phantom objecten aanmaken.
\item de rol van infrastructure master moet dus ook vervult worden door een server die geen global catalog is. Anders zijn er nooit verouderde objecten op het systeem. Indien alle controllers global catalog zijn, hebben ze allemaal een globale catalogus en is de rol niet meer van belang.
\end{itemize}
\end{enumerate}

\item \textbf{OM rollen die uniek moeten zijn binnen het forest}
\begin{itemize}
\item \textbf{Schema master} beheert alle bijgewerkte en gewijzigde gegeven voor het schema.
\item 1 server mag die maar doen om conflicten te vermijden
\item tijdelijk verlies van de schema master is niet merkbaar, tenzij er wijzigingen moeten doorgevoerd worden.
\end{itemize}

\item \textbf{Domain naming master}
\begin{itemize}
\item beheert het toevoegen en verwijderen van omeinen en applicatie partities  in het forest. 
\item het is de enige DC die de partitions container van de configuratiegegevens kan wijzigen.
\item moet een global catalog server zijn.
\item gebruikers ondervinden geen hinder als hij even offline is.
\end{itemize}

\clearpage
\item \textbf{Wie heeft welke rol?}
\begin{itemize}
\item wort bijgehouden in de fSMORoleOwner attributen van vijf verschillende obecten in verschillende partities. (p61)
\item een OM masters rol kan van andere domeincontrollers binnen het domein of het forest overgedragen worden.
\item indien je een nieuwe forest maakt worden alle single master rollen automatisch toegewezen aan de eerste domein controller in dat domein.
\end{itemize}
\end{itemize}

\item \textbf{Welke domein controller worden als ODC ingesteld?}
Alle nt5 controller beschkken over een equivalente wijzigbare kopie van alle AD partities vanaf windows 2008+ zijn er ook read only domain controllers.
\begin{itemize}
\item RODC beperken het replicatie verkeer maar in 1 richting
\item men kan dynamisch een RODC filtered attribute set configureren, een verzameling van alle kenmerken die niet naar een RODC gerepliceerd worden.
\item Men kan echter voor elke individuele RODC een password Replication Policy configureren die credential caching toestaat voor specifieke gebruikers en computers. Alle andere SPN objecten worden dan doorverwezen.
\item RODC kan ook functies vervullen van een globale catalogus of van een DNS server. In dat laatste geval bekomt men een secundaire nameserver.
\item een Operations master rol ondersteunen is niet mogelijk als RODC
\end{itemize}









\end{enumerate}
